\documentclass{jdf}

\begin{document}
\title{Mutual Monitoring in the Cloud}
\author{A.J. Stein \\ Georgia Institute of Technology \\ astein38@gatech.edu}

\maketitle
\thispagestyle{fancy}

\begin{abstract}
    Cloud computing infrastructure is essentially ubiquitous, but adoption is not without challenges. Cloud service providers must cater to customers in regulated sectors. Their use of popular cybersecurity frameworks create high barriers to entry. One barrier, often resulting in centralized bureaucracies, is the periodic monitoring of the provider's cybersecurity posture. By analyzing one prominent example, FedRAMP's Continuous Monitoring Program, this paper considers if such bureaucracies are the only valid solution. To refute this hypothesis, the paper presents an alternative architecture for multi-party monitoring of cloud services' cybersecurity posture, mutual monitoring.
\end{abstract}

\section{Introduction}

Cloud computing infrastructure is essentially ubiquitous, but adoption is not without challenges. Cloud service providers must cater to customers in regulated sectors, complying with cybersecurity frameworks that create high barriers to entry. One barrier is ongoing monitoring of the provider's cybersecurity posture, often resulting in centralized bureaucracies. FedRAMP oversees and documents a prominent example of such a program, the Continuous Monitoring Program \citeyear[p.~14]{fedramp_auth_playbook25}.

Are these bureaucracies an optimal solution, or a last resort that fails to keep pace with cloud technology as it proliferates and evolves? If they are a last resort, is there a better way? This paper presents an alternative, the mutual monitoring architecture, as a measurably more effective solution.

\subsection{Why Does This Problem Matter?}

The cybersecurity of cloud services poses many challenges, but the inefficiency of continuous monitoring has systemic impact on the economics and timely, accurate risk modeling for heavily interconnected, interdependent systems built on cloud services. FedRAMP is a highly visible and representative example that other regulatory frameworks emulate, so any improvement or optimization will yield significant improvement to cloud service adoption across regulated industries.

\subsection{Economic Impacts}

Although FedRAMP is a highly visible cloud security program, there is limited public data with details about costs and economic impact for providers, auditors, and customer agencies. However, industry estimates significant costs for all these stakeholders, even when considering global expenditure on cloud services.

Gartner estimates that global spending on cloud infrastructure in 2024 was \$595.7 billion dollars \citeyear{gartner24}. The think tank CSIS estimates that the United States government spent \$17 billion of its total \$130 billion dollar IT budget in 2024 on cloud services alone \citeyear[p.~1]{csis25}. Although federal agencies are not fully compliant with FedRAMP's requirements mandated in the FedRAMP Authorization Act, the long-term goal is maximal oversight over the cloud building blocks of this seventeen billion dollar investment. And continuous monitoring is a sizable component of this investment.

FedRAMP processes require specialized tools and staff for all stakeholders. Analysts at stackArmor estimate that a FedRAMP authorization costs a provider \$250,000 to \$750,000 dollars, and continuous monitoring support constitutes from \$100,000 to \$400,000 of that amount \citeyear{stackarmor24}. Given this conservative estimate, any improvement or optimization can benefit all stakeholders in reducing \$42,600,000 spent, but potentially a much larger sum.

\subsection{Cybersecurity Impacts}

Even with all this investment, the staff from cloud service providers, auditors, and agency customers experience strategic and operational bottlenecks for heavily interconnected cloud services, increasing ambiguity in a holistic view of cybersecurity posture in real-world composite systems for all parties involved, not only auditors. 

Firstly, a centralized review process finalized by a small number of FedRAMP staff constitutes a single point of failure. As FedRAMP documents, cloud providers, auditors, and agency customers must use a single, centralized wiki site, USDA's connect.gov, \footnote{This system is essentially the same as max.gov, which the Office of Management and Budget handed off to the Department of Agriculture subsequently, which FedRAMP had used in the years prior. USDA \hyperlink{https://www.fedramp.gov/2023-11-13-usda-connect-update-to-fedramp-stakeholders/}{rebranded the system in 2023} during the transition.} and coordinate out of band with FedRAMP staff for final review \citeyear[pp.~3,14]{fedramp_auth_playbook25}. Paradoxically, providers and auditors get no guarantees for the cybersecurity posture of this system where they store data for FedRAMP's reviewers. There is no mutual monitoring or assurance. Access to this data on connect.gov is manually coordinated on an ad hoc basis, hindering sharing between different agency staff who need FedRAMP data, and even those outside these agencies focused on other regulatory frameworks. They rely on reciprocity guarantees to justify the use of FedRAMP authorization and continuous monitoring, which is not particularly feasible in practical terms given restricted access to this data.

The impacts of manually curated data from FedRAMP's continuous monitoring extend beyond its stakeholders. Interrelated regulatory frameworks depend upon it. Given FedRAMP's rigorous review process, especially continuous monitoring, many providers and their auditors use artifacts from FedRAMP for equivalency, or reciprocity,as evidence for controls in other regulatory frameworks preferred by the defense \cite{dod_fedramp_memo23}, commercial \cite{orock21}, and finance sectors of the United States. Therefore, any optimization in FedRAMP's processes has second order effects on the quality, quantity, and speed of cloud security review methodologies across industry.

The impacts of manually curated data from FedRAMP's continuous monitoring extend beyond its stakeholders. Interrelated regulatory frameworks depend upon it. Given FedRAMP's rigorous review process, especially continuous monitoring, many providers and their auditors use artifacts from FedRAMP for equivalency, or reciprocity,as evidence for controls in other regulatory frameworks preferred by the defense, commercial, and finance sectors of the United States. Therefore, any optimization in FedRAMP's processes has second order effects on the quality, quantity, and speed of cloud security review methodologies across industry.

\subsection{Solution}

The focus of this paper is an alternative solution to centralized continuous monitoring as exemplified by FedRAMP, mutual monitoring. Mutual monitoring facilitates federated data services with ledgers\footnote{Many associate the term ``ledger" primarily with cryptocurrency and popular blockchain solutions, such as Bitcoin and Ethereum. In computing, a ledger is ``tamper-resistant shared distributed ledger composed of temporally ordered and publicly verifiable transactions." \cite{bashir22} Transparency service implementers and standards authors employ the same fundamental concept, but use the interchangeable term Append-only Log, which they define as ``a Statement Sequence comprising the entire registration history of the Transparency Service. To make the Append-only property verifiable and transparent." \cite{scitt25}. All are examples of distributed ledger technology.} of digitally signed data using an architecture popular for other security use cases, \hyperlink{https://transparency.dev}{transparency services}. The positives and negatives of FedRAMP's continuous monitoring model will inform its design. Operating such services can change the incentives, behavior, and thereby economics, of cloud service providers, auditors, and customers for true ``shared responsibility''\footnote{FedRAMP, like many cloud security programs, assert that ``[t]here is a shared security responsibility model when using cloud products. Cloud service providers (CSPs) and customers (agencies or leveraging CSPs) both assume important security roles and responsibilities to ensure data is protected within cloud environments.''\citeyear{fedramp_srm25} As practical as it sounds, there are many concerns and criticisms on how to meaningfully realize the shared responsibility model, which has direct implications on the current continuous monitoring process or mutual monitoring.} for cloud security monitoring. A new architecture should incentivize auditors to sell value-add analytics via these federated data services, potentially obsoleting centralized authorities for continuous monitoring like FedRAMP. To validate this hypothesis, I propose the list of deliverables below, in addition to the final report summarizing their outcome. 

To best explain the merits (and challenges) of mutual monitoring, the paper will provide an overview of past, present, and ongoing modernization of FedRAMP's continuous monitoring and overall processes. This context will inform the following section, that outlines the key elements of the proposed mutual monitoring architecture. And finally, the paper will conclude with a qualitative and quantitative evaluation of the solution, highlight key limitations, and identity future work to advance this solution.

\section{Background}

\subsection{Overview of Cloud Service Security Monitoring}

Despite the prominence of FedRAMP in cloud security inside and outside of government,\footnote{As ORock analysts note, FedRAMP is not required for customers outside of the federal government, but some companies recommend it for other regulated use cases nonetheless \citeyear{orock21}.} there is a body of work from different academic and industry experts with differing approaches to cloud security monitoring both before and during FedRAMP's expansion over the thirteen years. The following section discuss relevant highlights to current challenges to FedRAMP's continuous monitoring approach and the proposed mutual monitoring solution.

\subsubsection{Academic Research in Cloud Security Monitoring}

Over the last decade, academic researchers have affirmed the fundamentals of cloud deployment and security properties. Much literature uses the same taxonomy as Majumdar and his co-authors for cloud security auditing as reactive, intercept-and-check, or proactive \citeyear[pp.~9-13]{majumdar19}. Nonetheless, most research does not focus on transparency services or similar solutions to multi-party cloud security auditing or monitoring.\footnote{Both academia and industry uses the terms auditing and monitoring to practically mean the same thing.}

In their survey, Ramachandra and his colleagues identify a key property to security and risk exposure of cloud infrastructures past and present: the two most important aspects in determining impact and exposure to vulnerabilities is the choice of deployment (e.g. public or private) and delivery model (e.g. Infrastructure-as-a-Service (IaaS); Platform-as-a-Service (PaaS); Software-as-a-Service (Saas)) \citeyear[p.~468]{ramachandra17}. This research focuses primarily on public deployment for the various delivery models. According to this research, this subset experiences heightened security challenges due to a large customer footprint, management of publicly available resources, and a multitude of external factors outside of their immediate control, including legislation and data protection laws \cite[p.468]{ramachandra17}. The matrix of cloud deployment models and security responsibility still holds true today, in that customers bare more responsibility with IaaS to shape their own infrastructure accordingly. Conversely, PaaS to a great extent, and SaaS to the greatest extent, burden the cloud providers with securing the system, not the customer. \cite[p.~469]{ramachandra17}. Interestingly, in this 2017 survey there is no mention on monitoring, coordination, or transparency about security posture with well-informed customers as an impact or challenge in current literature and practice. The paper does not list them as defensible controls or counter-measures either.

Similarly, older surveys of cloud monitoring (not just specifically to security), such as Aceto and his colleagues, do not identify these themes or trends relevant to security monitoring for multi-tenant cloud customers \citeyear{aceto13}.

Hakani and Mann have a more current survey for cloud security mechanisms, confirming deployment types and models have not much changed, but expounding more on updated detailed security threats, cloud data security, cloud firewalling, and key management approaches \citeyear{hakani22}. Although there is hardly any discussion of research on monitoring and coordination between cloud provider, auditor, or customer, this survey does allude to their absence as a significant challenge: ``Whilst using cloud technology, both customers and providers face several security concerns and issues. Such issues may make it harder for customers as well as suppliers to believe one another.'' \citeyear[p.~475]{hakani22}

Although general surveys did not focus on these particular challenges and solution, there is a wide variety of solutions to security monitoring TODO REWORD. Majumdar and his colleagues advocate for a proactive auditing with a system proven by formal methods to detect security violations from events and recycles verification results to restore policies \citeyear[p.~2518]{majumdar22}. The design of Aldribi and his team employs underlying hardware isolation to empower customers to configure and monitor systems accordingly in complex multi-tenant environments \citeyear{aldribi15}. Carvallo and other researchers present a design for a comprehensive security assurance platform with network, system, and application monitoring sensors for internal reporting \citeyear{carvallo17}. Torkura and his colleagues have their own novel solution for monitoring misconfigurations with their CSBAuditor using transition analysis and the reconciler pattern \citeyear{torkura21}.

As promising as all of these solutions are, whether proactive, intercept-and-check, or reactive, no solution takes a similar approach to mutual monitoring.

\subsection{Cybersecurity Frameworks and Cloud Security Monitoring}

The previous section identifies a wide variety of research into cloud security monitoring, but without explaining why there is practical industry interest in monitoring. A primary reason is that common cybersecurity frameworks, used by both cloud service providers and their customers, recommend or require periodic monitoring of their infrastructure. This section will identify those requirements in the most common cybersecurity frameworks.

\subsubsection{NIST Risk Management Framework}

NIST's Risk Management Framework (RMF), as the foundation FedRAMP used for its own design, has a very straightforward requirement for continuous monitoring. FedRAMP tailors the RMF (defined in NIST Special Publication 800-37) and its catalog of controls (defined in NIST Special Publication 800-53) for cloud services, whereas government requires RMF for many other systems, not just the cloud. Most tailored uses of RMF mandate implementation of control CA-7, requiring an organization to establish an continuous monitoring program \citeyear[pp.~90-91]{sp80053r5}.

\subsubsection{ISO/IEC 27001:2022}

The International Organization for Standardization (ISO) is a voluntary standards body that promulgates standards for many nations, unlike the previous examples that are predominantly focused on the United States. ISO 27001:2022, their framework for building information security management systems, has control Appendix A 8.16, which recommends continuous monitoring \cite{iso27001_22}

\subsubsection{Cloud Security Alliance Cloud Controls Matrix}

The Cloud Security Alliance (CSA) is a reputable organization that promulgates security guidance for cloud service providers, including their own cybersecurity framework, the Cloud Control Matrix (CCM). The CSA maintains a registry, STAR, for certified providers that meet different maturity levels for their implementation of the CCM controls. In 2024, CSA published STAR Level 3, requires continuous monitoring for this highest maturity level \citeyear{csa_starl3_21}.

\subsubsection{CIS Critical Security Control}

The Center for Internet Security maintains a popular cybersecurity framework for simple best practices that apply to wide spectrum of companies. One control in their Critical Security Controls framework is CSC-7, which requires continuous vulnerability management \citeyear{csc18}.

\subsection{FedRAMP History and Continuous Monitoring}

\subsubsection{History}

\subsubsection{Continuous Monitoring}

\subsection{Transparency Services for Other Use Cases}

\subsection{Solution}

\subsection{Mutual Monitoring Transparency Service}

\subsection{Quantitative Framework}

\section{Evaluation}

\subsection{Method and Results}

\subsection{Limitations and Future Work}

\subsection{Conclusion}

\section{Appendix}

\bibliographystyle{apacite}
\bibliography{references.bib}

\end{document}
