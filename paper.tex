\documentclass{jdf}

\begin{document}
\title{Mutual Monitoring in the Cloud}
\author{A.J. Stein \\ Georgia Institute of Technology \\ astein38@gatech.edu}

\maketitle
\thispagestyle{fancy}

\begin{abstract}
    Cloud computing infrastructure is essentially ubiquitous, but adoption is not without challenges. Cloud service providers must cater to customers in regulated sectors. Their use of popular cybersecurity frameworks create high barriers to entry. One barrier, often resulting in centralized bureaucracies, is the periodic monitoring of the provider's cybersecurity posture. By analyzing one prominent example, FedRAMP's Continuous Monitoring Program, this paper considers if such bureaucracies are the only valid solution. To refute this hypothesis, the paper presents an alternative architecture for multi-party monitoring of cloud services' cybersecurity posture, Mutual Monitoring.
\end{abstract}

\section{Introduction}

\section{Overview of Cloud Service Security Monitoring}

\section{FedRAMP Continuous Monitoring}

\section{Transparency Services for Other Use Cases}

\section{Mutual Monitoring Transparency Service}

\section{Quantitative Framework}

\section{Evaluation}

\section{Limitations and Future Work}

\section{Conclusion}

\section{Appendix}

\bibliographystyle{apacite}
\bibliography{references.bib}

\end{document}
